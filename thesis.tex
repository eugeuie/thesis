\documentclass[12pt, a4paper, oneside]{article}
\usepackage[utf8]{inputenc}
\usepackage[english, russian]{babel}
\usepackage[T1, T2A]{fontenc}
\usepackage{titlesec}
\usepackage{hyperref}
\usepackage{tikz}
\usepackage[left=4cm,right=2cm,top=4cm,bottom=4cm,bindingoffset=0cm]{geometry}
\tolerance=10000 % разреженность строки
\hypersetup{
    colorlinks=true,
    linkcolor=black,
    urlcolor=blue,
    citecolor=blue,
}

\title{Thesis}
\author{Evgenia Abdullaeva}
\date{May 2021}

\begin{document}
\begin{titlepage}
    \begin{center}
        Филиал Московского Государственного Университета\\
        имени М.В. Ломоносова в городе Ташкенте\\
        \vspace{0.5cm}
        Факультет прикладной математики и информатики\\
        Кафедра прикладной математики и информатики\\
        \vfill
        Абдуллаева Евгения Гасановна\\
        \vfill
        \textbf{ВЫПУСКНАЯ КВАЛИФИКАЦИОННАЯ РАБОТА\\
        на тему: «Разработка модулей оповещения и статистики для системы дистанционного обучения»}\\
        \vspace{0.5cm}
        по направлению 01.03.02 «Прикладная математика и информатика»
    \end{center}
    \vfill
    \begin{flushleft}
        Выпускная квалификационная работа рассмотрена и рекомендована к защите\\
        \vspace{0.5cm}
        Руководитель Филиала МГУ имени М.В. Ломоносова в г. Ташкенте\\
        к.ф.-м.н., доцент Строгалов Александр Сергеевич\\
        \vspace{0.5cm}
        Научный руководитель\\
        к.ф.-м.н., в.н.с. Алисейчик Павел Александрович
    \end{flushleft}
    \vfill
    \begin{flushright}
        «$\rule{1cm}{0.15mm}$» $\rule{3cm}{0.15mm}$ 2021 г.
    \end{flushright}
    \vfill
    \begin{center}
        Ташкент 2021 г.
    \end{center}
\end{titlepage}

\begin{abstract}
Текст аннотации
\end{abstract}
\selectlanguage{english}
\begin{abstract}
    Abstract text
\end{abstract}
\selectlanguage{russian}
\setcounter{page}{2}
\newpage

\tableofcontents
\newpage

\section{Введение}
\paragraph{} % Постановка проблемы, цель работы
Постановка проблемы, цель работы
\paragraph{} % Актуальность
Актуальность
\newpage

\section{Общая структура и организация работы}
\paragraph{}
Настоящая работа содержит две основные части ...
\paragraph{}
Немного о Django
\newpage

\section{Модуль оповещения}
\newpage

\section{Модуль статистики}
\newpage

\section{Раздел помощи}
\newpage

\section{Заключение}
\paragraph{}
Текст заключения
\newpage

\section{Приложения}
\newpage

\section{Список использованных источников и литературы}
\begingroup
\renewcommand{\section}[2]{}%
\begin{thebibliography}{20}
    \bibitem{msu-article-1}
    Моделирование процесса обучения / В. Б. Кудрявцев, П. А. Алисейчик, К. Вашик, Ж. Кнап, А. С. Строгалов, С. Г. Шеховцов // Интеллектуальные системы. – 2006, т. 10, вып. 1-4, стр. 189-270.
    \bibitem{msu-article-2}
    О дистанционном образовании – пример реализации и перспективы / П. А. Алисейчик, А. С. Строгалов, Р. А. Бекташев // Интеллектуальные системы. Теория и приложения. – 2016, т. 20, вып. 3, стр. 127-133.
    \bibitem{msu-contest}
    Репозиторий проекта «МГУ Контест»\\
    \url{https://github.com/ruslanbektashev/contest}
    \bibitem{msu-contest-site}
    Сайт «МГУ Контест»\\
    \url{https://contest.msu-dev.ru/}
    \bibitem{django}
    Документация фреймворка Django\\
    \url{https://docs.djangoproject.com/en/3.0}
    \bibitem{python}
    Документация языка Python\\
    \url{https://docs.python.org/3}
    \bibitem{chartjs}
    Документация инструмента Chart.js\\
    \url{https://www.chartjs.org/docs/latest/}
\end{thebibliography}
\endgroup

\end{document}
